\newif\ifen
\newif\ifes
\newif\iffr
\newcommand{\fr}[1]{\iffr#1 \fi}
\newcommand{\En}[1]{\ifen#1\fi}
\newcommand{\Es}[1]{\ifes#1\fi}
\estrue
\documentclass[shownotes,aspectratio=169]{beamer}

\usepackage{siunitx}
\input{../../auxiliar/tex/diapo_encabezado.tex}
\input{../../auxiliar/tex/tikzlibrarybayesnet.code.tex}
 \mode<presentation>
 {
 %   \usetheme{Madrid}      % or try Darmstadt, Madrid, Warsaw, ...
 %   \usecolortheme{default} % or try albatross, beaver, crane, ...
 %   \usefonttheme{serif}  % or try serif, structurebold, ...
  \usetheme{Antibes}
  \setbeamertemplate{navigation symbols}{}
 }
\estrue
\usepackage{todonotes}
\setbeameroption{show notes}
%
\newcommand{\gray}{\color{black!55}}
\usepackage{ulem} % sout
\usepackage{mdframed}
\usepackage{comment}
\usepackage{listings}
\lstset{
  aboveskip=3mm,
  belowskip=3mm,
  showstringspaces=true,
  columns=flexible,
  basicstyle={\ttfamily},
  breaklines=true,
  breakatwhitespace=true,
  tabsize=4,
  showlines=true
}


\begin{document}

\color{black!85}
\large
%
% \begin{frame}[plain,noframenumbering]
%
%
% \begin{textblock}{160}(0,0)
% \includegraphics[width=1\textwidth]{../../auxiliar/static/deforestacion}
% \end{textblock}
%
% \begin{textblock}{80}(18,9)
% \textcolor{black!15}{\fontsize{44}{55}\selectfont Verdades}
% \end{textblock}
%
% \begin{textblock}{47}(85,70)
% \centering \textcolor{black!15}{{\fontsize{52}{65}\selectfont Empíricas}}
% \end{textblock}
%
% \begin{textblock}{80}(100,28)
% \LARGE  \textcolor{black!15}{\rotatebox[origin=tr]{-3}{\scalebox{9}{\scalebox{1}[-1]{$p$}}}}
% \end{textblock}
%
% \begin{textblock}{80}(66,43)
% \LARGE  \textcolor{black!15}{\scalebox{6}{$=$}}
% \end{textblock}
%
% \begin{textblock}{80}(36,29)
% \LARGE  \textcolor{black!15}{\scalebox{9}{$p$}}
% \end{textblock}
%
% %
% %
% % \begin{textblock}{160}(01,81)
% % \footnotesize \textcolor{black!5}{\textbf{\small Seminario ``Acuerdos intersubjetivos''\\
% % Comunidad Bayesiana Plurinacional} \\}
% % \end{textblock}
%
% \end{frame}

%%%%%%%%%%%%%%%%%%%%%%%%%%%%%%%%%%%%%%%%%

\begin{frame}[plain,noframenumbering]
\begin{textblock}{160}(0,43)
\includegraphics[width=1\textwidth]{../../auxiliar/static/modelosGraficos}
\end{textblock}


\begin{textblock}{160}(4,4)
\LARGE \textcolor{black!85}{\fontsize{22}{0}\selectfont \textbf{Evaluación de modelo}}
\end{textblock}
% \begin{textblock}{160}(4,12)
% \LARGE \textcolor{black!85}{\fontsize{22}{0}\selectfont \textbf{algoritmos de inferencia}}
% \end{textblock}


\begin{textblock}{55}[0,0](72,23)
\begin{turn}{0}
\parbox{10cm}{\sloppy\setlength\parfillskip{0pt}
\textcolor{black!85}{Unidad 1} \\
\small\textcolor{black!85}{Acuerdos intersubjetivos en contextos de incertidumbre.} \\
\small\textcolor{black!85}{Especificación gráfica de modelos causales. Evaluación} \\
\small\textcolor{black!85}{de modelos causales. La emergencia del sobreajuste y el} \\
\small\textcolor{black!85}{balance natural de las reglas de la probabilidad.} \\
}
\end{turn}
\end{textblock}

\end{frame}

\begin{frame}[plain]
\begin{textblock}{160}(00,04)
\centering
\LARGE Verdad
\end{textblock}
\vspace{1.3cm} \large

\centering

 La ciencia tiene pretensión de verdad, de alcanzar\\

\textbf{acuerdos intersubjetivos que valgan para todas las personas}

\vspace{0.7cm}

\pause

 \large Ciencias formales  \\
 \large  Sistemas axiomáticos \textbf{cerrados} sin incertidumbre\\

 \vspace{0.3cm}

  \pause

 \large Ciencias con datos  \\
\large Sistemas naturales \textbf{abiertos} con incertidumbre

\pause
\vspace{0.6cm}

\Large

¿Qué es una verdad en \\ contextos de incertidumbre?
%
% \pause
% \vspace{0.2cm}
%
%
% Sí. Podemos evitar mentir.

\end{frame}


\begin{frame}[plain]
\begin{textblock}{160}(00,04)
\centering
\only<1>{\LARGE ¿Todo vale lo mismo?}\only<2>{\Large Ouch!.. Me estás pisando el cuello!}\only<3>{\Large Bueno, es un punto de vista. Se podría decir que me querés hacer tropezar.}\only<4>{\Large En la posmodernidad ya no hay objetividad, toda historia es igualmente válida}\only<5->{\Large \ Pero me sigues pisando el cuello!\hfill Nunca fuiste a la universidad \ } \\
\end{textblock}
\vspace{1cm} \large


\only<2>{
\begin{textblock}{160}(0,14) \centering
\includegraphics[width=0.45\textwidth, page={6}]{../../auxiliar/static/sidewalk_bubblegum_1997_1}
\end{textblock}}
 \only<3>{
\begin{textblock}{160}(0,14) \centering
\includegraphics[width=0.45\textwidth, page={6}]{../../auxiliar/static/sidewalk_bubblegum_1997_2}
\end{textblock}}
\only<4>{
\begin{textblock}{160}(0,14) \centering
\includegraphics[width=0.45\textwidth, page={6}]{../../auxiliar/static/sidewalk_bubblegum_1997_3}
\end{textblock}}
\only<5>{
\begin{textblock}{160}(0,14) \centering
\includegraphics[width=0.45\textwidth, page={6}]{../../auxiliar/static/sidewalk_bubblegum_1997_4}
\end{textblock}}

\end{frame}

\begin{frame}[plain]
\only<2->{
\begin{textblock}{160}(0,4) \centering
\Large Al menos sabemos cómo no mentir
\end{textblock}
}
\only<1>{
\begin{textblock}{160}(0,34) \centering
\LARGE ¿Puede haber una verdad si justamente \\ tenemos incertidumbre sobre su verdadero valor?
\end{textblock}
}
\vspace{2cm}

\pause
\pause

\Large

\centering

$\bullet$ No afirmar más de lo que se conoce \pause

$\bullet$ Sin ocultar aquello que sí se conoce

\pause \centering \vspace{1cm}

\Large

\textbf{¿Cómo exactamente?}


\end{frame}


\begin{frame}[plain]
 \begin{textblock}{160}(0,4)
 \centering \LARGE \only<2>{Certeza absoluta}\only<3>{Distribución de creencias \\ }\only<4-5>{Distribución de creencias \\ \Large Honesta }
 \only<6->{¿Cómo preservamos los acuerdos intersubjetivos?\\}
\end{textblock}
\vspace{1.5cm}
\centering


\only<1>{
\begin{textblock}{160}(0,62)
\Large Detrás de una de estas caja hay un regalo. \\[0.1cm]

\large ¿Dónde está el regalo?
\end{textblock}
}

\only<1>{
\begin{textblock}{160}(0,28)
 \scalebox{1.1}{
\tikz{ %
         \node[factor, minimum size=1cm] (p1) {} ;
         \node[factor, minimum size=1cm, xshift=1.5cm] (p2) {} ;
         \node[factor, minimum size=1cm, xshift=3cm] (p3) {} ;


         \node[const, above=of p1, yshift=0.1cm] (np1) {\Large $?$};
         \node[const, above=of p2, yshift=0.1cm] (np2) {\Large $?$};
         \node[const, above=of p3, yshift=0.1cm] (np3) {\Large $?$};
         }
}
\end{textblock}
}

\only<2>{
\begin{textblock}{160}(0,28)
 \scalebox{1.1}{
\tikz{ %
         \node[factor, minimum size=1cm] (p1) {} ;
         \node[factor, minimum size=1cm, xshift=1.5cm] (p2) {} ;
         \node[factor, minimum size=1cm, xshift=3cm] (p3) {} ;


         \node[const, above=of p1, yshift=0.125cm] (np1) {\Large $0$};
         \node[const, above=of p2, yshift=0.125cm] (np2) {\Large $1$};
         \node[const, above=of p3, yshift=0.125cm] (np3) {\Large $0$};
         }
}
\end{textblock}
}

\only<3>{
\begin{textblock}{160}(0,28)
 \scalebox{1.1}{
\tikz{ %
         \node[factor, minimum size=1cm] (p1) {} ;
         \node[factor, minimum size=1cm, xshift=1.5cm] (p2) {} ;
         \node[factor, minimum size=1cm, xshift=3cm] (p3) {} ;


         \node[const, above=of p1, yshift=-0.05cm] (np1) {\Large $1/10$};
         \node[const, above=of p2, yshift=-0.05cm] (np2) {\Large $8/10$};
         \node[const, above=of p3, yshift=-0.05cm] (np3) {\Large $1/10$};
         }
}
\end{textblock}
}


\only<4-5>{
\begin{textblock}{160}(0,28)
 \scalebox{1.1}{
\tikz{ %
         \node[factor, minimum size=1cm] (p1) {} ;
         \node[factor, minimum size=1cm, xshift=1.5cm] (p2) {} ;
         \node[factor, minimum size=1cm, xshift=3cm] (p3) {} ;


         \node[const, above=of p1, yshift=-0.05cm] (np1) {\Large $1/3$};
         \node[const, above=of p2, yshift=-0.05cm] (np2) {\Large $1/3$};
         \node[const, above=of p3, yshift=-0.05cm] (np3) {\Large $1/3$};
         }
}
\end{textblock}
}

\only<5>{
\begin{textblock}{150}(10,64)   \centering \Large
Acuerdo intersubjetivo (no mentir)\\[0.1cm]
\large 1. \textbf{Maximizar incertidumbre} (dividiendo en partes iguales) \\
\large 2. \textbf{Dada la información disponible} (no asignamos creencia por fuera de las cajas)

\end{textblock}
}

\only<6->{
\begin{textblock}{160}(0,28)
 \scalebox{1.1}{
\tikz{ %
         \node[factor, minimum size=1cm] (p1) {} ;
         \node[det, minimum size=1cm, xshift=1.5cm] (p2) {\includegraphics[width=0.03\textwidth]{../../auxiliar/static/dedo.png}} ;
         \node[factor, minimum size=1cm, xshift=3cm] (p3) {} ;


         \node[const, above=of p1, yshift=-0.05cm] (np1) {\Large $\phantom{/}?\phantom{/}$};
         \node[const, above=of p2, yshift=-0.05cm] (np2) {\Large $\phantom{/}0\phantom{/}$};
         \node[const, above=of p3, yshift=-0.05cm] (np3) {\Large $\phantom{/}?\phantom{/}$};
         }
}
\end{textblock}
}



\end{frame}




\begin{frame}[plain]
\begin{textblock}{160}(0,4)
 \centering \LARGE Modelos causales \\
\end{textblock}
\vspace{1cm}


\begin{textblock}{160}(8,22)
%\onslide<2->{Modelo gráfico} \\ \vspace{0.3cm}
 \tikz{
    \node[latent,] (r) {\includegraphics[width=0.06\textwidth]{../../auxiliar/static/regalo.png}} ;
    \node[const,above=of r, xshift=-0.2cm, yshift=0.3cm] (titulo) {\Large Modelo gráfico} ;
    \node[const,left=of r] (nr) {Regalo: \Large $r$\,} ;

    \onslide<2->{
    \node[latent, below=of r] (d) {\includegraphics[width=0.05\textwidth]{../../auxiliar/static/dedo.png}} ;
    \node[const, left=of d] (nd) {Pista: \Large $s$\,} ;
    \node[const, below=of d, yshift=-0.2cm] (c) {$(s \neq r)$};

    \edge {r} {d};
    }
}
\end{textblock}

\only<1-2>{
\begin{textblock}{160}(65,33)
\scalebox{1.5}{
\tikz{
    \node[factor, minimum size=1cm] (p1) {} ;
    \node[factor, minimum size=1cm, xshift=1.5cm] (p2) {} ;
    \node[factor, minimum size=1cm, xshift=3cm] (p3) {} ;

    \node[const, above=of p1, yshift=.15cm] (fp1) {$1/3$};
    \node[const, above=of p2, yshift=.15cm] (fp2) {$1/3$};
    \node[const, above=of p3, yshift=.15cm] (fp3) {$1/3$};
    \node[const, below=of p2, yshift=-.10cm, xshift=0.3cm] (dedo) {};

    \node[invisible, xshift=4.75cm] (s-dist) {};
    \node[invisible, yshift=-1cm] (s-dist) {};
    \node[invisible, yshift=1.2cm] (s-dist) {};
    }
}
\end{textblock}
}

\only<3>{
\begin{textblock}{160}(65,33)
\scalebox{1.5}{
\tikz{ %

         \node[factor, minimum size=1cm] (p1) {} ;
         \node[det, minimum size=1cm, xshift=1.5cm] (p2) {\includegraphics[width=0.03\textwidth]{../../auxiliar/static/dedo.png}} ;
         \node[factor, minimum size=1cm, xshift=3cm] (p3) {} ;
%
%
         \node[const, above=of p1, yshift=.15cm] (fp1) {$?$};
         \node[const, above=of p2, yshift=.15cm] (fp2) {$0$};
         \node[const, above=of p3, yshift=.15cm] (fp3) {$?$};
         \node[const, below=of p2, yshift=-.10cm, xshift=0.3cm] (dedo) {};

%         \node[const, above=of p2, xshift=.8cm, yshift=.15cm] (fp3) {$66\%$};
%
         \node[invisible, xshift=4.75cm] (s-dist) {};
         \node[invisible, yshift=-1cm] (s-dist) {};
         \node[invisible, yshift=1.2cm] (s-dist) {};
%
%         \plate[color=red] {no} {(p1)} {}; %
%         \plate {si} {(p2)(p3)} {}; %

        }
}
\end{textblock}
}

\end{frame}

\begin{frame}[plain]
\begin{textblock}{160}(0,4)
 \centering \LARGE Modelos causales\\
 \Large Máxima incertidumbre dado el modelo \\
\end{textblock}
\vspace{1cm}
\vspace{1cm}


\only<1-3>{
\begin{textblock}{160}(8,22)
%\onslide<2->{Modelo gráfico} \\ \vspace{0.3cm}
 \tikz{
    \node[latent,] (r) {\includegraphics[width=0.06\textwidth]{../../auxiliar/static/regalo.png}} ;
    \node[const,above=of r, xshift=-0.2cm, yshift=0.3cm] (titulo) {\Large {Modelo gráfico}} ;
    \node[const,left=of r] (nr) {Regalo: \Large $r$\,} ;

    \node[latent, below=of r] (d) {\includegraphics[width=0.05\textwidth]{../../auxiliar/static/dedo.png}} ;
    \node[const, left=of d] (nd) {Pista: \Large $s$\,} ;
    \node[const, below=of d, yshift=-0.2cm] (c) {$(s \neq r)$};

    \edge {r} {d};

}
\end{textblock}
}


\only<1->{
\begin{textblock}{80}(60,20) \centering
\scalebox{1.1}{
\tikz{
\onslide<1->{
\node[latent, draw=white, yshift=0.6cm] (b0) {$ 1$};

\node[latent,below=of b0,yshift=0.6cm, xshift=-3cm] (r1) {$r_1$};
\node[latent,below=of b0,yshift=0.6cm] (r2) {$r_2$};
\node[latent,below=of b0,yshift=0.6cm, xshift=3cm] (r3) {$r_3$};

\node[latent, below=of r1, draw=white, yshift=0.6cm] (br1) {$\frac{1}{3}$};
\node[latent, below=of r2, draw=white, yshift=0.6cm] (br2) {$\frac{1}{3}$};
\node[latent, below=of r3, draw=white, yshift=0.6cm] (br3) {$\frac{1}{3}$};
}
\onslide<2->{
\node[latent,below=of br1,yshift=0.6cm, xshift=-0.7cm] (r1d2) {$s_2$};
\node[latent,below=of br1,yshift=0.6cm, xshift=0.7cm] (r1d3) {$s_3$};

\node[latent,below=of r1d2,yshift=0.6cm,draw=white] (br1d2) {$\frac{1}{3}\frac{1}{2}$};
\node[latent,below=of r1d3,yshift=0.6cm, draw=white] (br1d3) {$\frac{1}{3}\frac{1}{2}$};
}
\onslide<3->{
\node[latent,below=of br2,yshift=0.6cm, xshift=-0.7cm] (r2d1) {$s_1$};
\node[latent,below=of br2,yshift=0.6cm, xshift=0.7cm] (r2d3) {$s_3$};
\node[latent,below=of br3,yshift=0.6cm, xshift=-0.7cm] (r3d1) {$s_1$};
\node[latent,below=of br3,yshift=0.6cm, xshift=0.7cm] (r3d2) {$s_2$};

\node[latent,below=of r2d1,yshift=0.6cm, draw=white] (br2d1) {$\frac{1}{3}\frac{1}{2}$};
\node[latent,below=of r2d3,yshift=0.6cm,draw=white] (br2d3) {$\frac{1}{3}\frac{1}{2}$};
\node[latent,below=of r3d1,yshift=0.6cm, draw=white] (br3d1) {$\frac{1}{3}\frac{1}{2}$};
\node[latent,below=of r3d2,yshift=0.6cm,draw=white] (br3d2) {$\frac{1}{3}\frac{1}{2}$};
}
\onslide<1->{
\edge[-] {b0} {r1,r2,r3};
\edge[-] {r1} {br1};
\edge[-] {r2} {br2};
\edge[-] {r3} {br3};
}
\onslide<2->{
\edge[-] {br1} {r1d2,r1d3};
\edge[-] {r1d2} {br1d2};
\edge[-] {r1d3} {br1d3};
}
\onslide<3->{
\edge[-] {br2} {r2d1, r2d3};
\edge[-] {br3} {r3d1,r3d2};
\edge[-] {r2d1} {br2d1};
\edge[-] {r2d3} {br2d3};
\edge[-] {r3d1} {br3d1};
\edge[-] {r3d2} {br3d2};
}
}
}
\end{textblock}
}


\only<4->{
 \begin{textblock}{65}(0,24)
  \centering
  Creencia$(r,s|\text{Modelo})$ \\ \vspace{0.3cm}
 \begin{tabular}{c|c|c|c||c} \setlength\tabcolsep{0.4cm}
        & \, $r_1$ \, &  \, $r_2$ \, & \, $r_3$ \, & \\ \hline
  $s_1$  & \onslide<5->{$0$} & \onslide<6->{$1/6$} & \onslide<6->{$1/6$} & \\ \hline
  $s_2$  & \onslide<7->{$1/6$} & \onslide<7->{$0$} & \onslide<7->{$1/6$} &  \\ \hline
       $s_3$ & \onslide<8->{$1/6$} & \onslide<8->{$1/6$} & \onslide<8->{$0$} &  \\ \hline \hline
              & & &  & \\
\end{tabular}
\end{textblock}
}



\only<9->{
 \begin{textblock}{60}(0,65) \centering
Creencia conjunta\\

intersubjetiva inicial
 \end{textblock}
 }
\end{frame}



\begin{frame}[plain]
 \begin{textblock}{160}(0,4)
 \centering \Large\hspace{1.4cm}Máxima incertidumbre dado el modelo \only<1-5>{\phantom}{y el dato}
 \end{textblock}

\vspace{1cm}

 \begin{textblock}{160}(0,15)
  \centering
  $\overbrace{\text{Creencia}(r,s|\text{M})}^{\text{\scriptsize De ambas variables}}$ \\ \vspace{0.3cm}
 \begin{tabular}{c|c|c|c||c} \setlength\tabcolsep{0.4cm}
     $\phantom{\bm{s_2}}$   & \, $r_1$ \, &  \, $\only<2>{\gray}r_2$ \, & \, $\only<2>{\gray}r_3$ \, &  \phantom{\bm{$1/3$}} \\ \hline
  $\only<5>{\gray}s_1$ & $\only<5>{\gray}0$ & $\only<2,5>{\gray}1/6$ & $\only<2,5>{\gray}1/6$ & \onslide<4->{$\only<5>{\gray}1/3$} \\ \hline
  $\only<5>{\bm}{s_2}$ & $1/6$ & $\only<2>{\gray}0$ & $\only<2>{\gray}1/6$ & \onslide<4->{$1/3$} \\ \hline
  $\only<5>{\gray}s_3$ & $\only<5>{\gray}1/6$ & $\only<2,5>{\gray}1/6$ & $\only<2,5>{\gray}0$ & \onslide<4->{$\only<5>{\gray}1/3$} \\ \hline \hline
        & \onslide<3->{$\only<5>{\gray}1/3$} & \onslide<3->{$\only<5>{\gray}1/3$} & \onslide<3->{$\only<5>{\gray}1/3$} &  \\
\end{tabular}

\vspace{0.3cm}

\onslide<2->{
\begin{align*}
 \text{Creencia}(r|\text{M}) = \onslide<3->{\sum_s \text{Creencia}(r,s|\text{M})}
\end{align*}
}
\vspace{-0.5cm}
\onslide<4->{
\begin{align*}
 \text{Creencia}(s|\text{M}) = \sum_r \text{Creencia}(r,s|\text{M})
\end{align*}
}
\end{textblock}

\end{frame}


\begin{frame}[plain]
 \begin{textblock}{160}(0,4)
 \centering \Large\hspace{1.4cm}Máxima incertidumbre dado el modelo y el dato
 \end{textblock}

\vspace{1cm}

\only<1->{
 \begin{textblock}{160}(0,15)
  \centering
  \only<1-2>{$\overbrace{\text{Creencia}(r,s_2|\text{M})}^{\text{\scriptsize De ambas variables}}$}\only<3->{$\overbrace{\text{Creencia}(r|s_2,\text{M})}^{\text{\scriptsize De ambas variables}}$} \\ \vspace{0.3cm}
 \begin{tabular}{c|c|c|c||c} \setlength\tabcolsep{0.4cm}
        $\phantom{\bm{s_2}}$ & \, $r_1$ \, &  \, $r_2$ \, & \, $r_3$ \, &  \phantom{\bm{$1/3$}} \\ \hline
  &  &  &  & \\ \hline
  $\bm{s_2}$ & \only<1-2>{$1/6$}\only<3>{$\frac{1}{6}/\frac{1}{3}$}\only<4->{$1/2$} & $0$ & \only<1-2>{$1/6$}\only<3>{$\frac{1}{6}/\frac{1}{3}$}\only<4->{$1/2$} & \only<1>{$1/3$}\only<2>{{$\bm{1/3}$}}\only<3>{$\frac{1}{3}/\frac{1}{3}$}\only<4->{$1$} \\ \hline
  &  &  & &  \\
\end{tabular}
\end{textblock}
}

\only<1>{
\begin{textblock}{160}(0,58)
\begin{equation*}
\ \phantom{\underbrace{\text{Creencia}(r|s_2,\text{M})}_{\text{Nueva creencia}} =} \hfrac{\overbrace{\text{Creencia}(r, s_2|\text{M})}^{\text{Creencia compatible}}}{\phantom{\underbrace{\text{Creencia}(s_2|\text{M})}_{\text{Creencia total que compatible}}}}
\end{equation*}
\end{textblock}
}

\only<2>{
\begin{textblock}{160}(0,58)
\begin{equation*}
\ \phantom{\underbrace{\text{Creencia}(r|s_2,\text{M})}_{\text{Nueva creencia}} =}\hfrac{\overbrace{\text{Creencia}(r, s_2|\text{M})}^{\text{Creencia compatible}}}{\underbrace{\text{Creencia}(s_2|\text{M})}_{\text{Creencia total que compatible}}}
\end{equation*}
\end{textblock}
}


\only<3-4>{
\begin{textblock}{160}(0,58)
\begin{equation*}
\underbrace{\text{Creencia}(r|s_2,\text{M})}_{\text{Nueva creencia}} = \frac{\overbrace{\text{Creencia}(r, s_2|\text{M})}^{\text{Creencia compatible}}}{\underbrace{\text{Creencia}(s_2|\text{M})}_{\text{Creencia total que compatible}}}
\end{equation*}
\end{textblock}
}


\only<5->{
\begin{textblock}{160}(7,57)
\centering
\scalebox{1.2}{
\tikz{ %

         \node[factor, minimum size=1cm] (p1) {} ;
         \node[det, minimum size=1cm, xshift=1.5cm] (p2) {\includegraphics[width=0.03\textwidth]{../../auxiliar/static/dedo.png}} ;
         \node[factor, minimum size=1cm, xshift=3cm] (p3) {} ;

         \node[const, above=of p1, yshift=.15cm] (fp1) {$1/2$};
         \node[const, above=of p2, yshift=.15cm] (fp2) {$0$};
         \node[const, above=of p3, yshift=.15cm] (fp3) {$1/2$};
         \node[const, below=of p2, yshift=-.10cm, xshift=0.3cm] (dedo) {};

         \node[invisible, xshift=4.75cm] (s-dist) {};
         \node[invisible, yshift=-1cm] (s-dist) {};
         \node[invisible, yshift=1.2cm] (s-dist) {};

        }
}
\end{textblock}
}

\end{frame}

{
\setbeamercolor{background canvas}{bg=orange!20}
\begin{frame}[plain]
\begin{textblock}{160}(0,4)
\centering \LARGE Las reglas de la probabilidad
\end{textblock}
\onslide<1>{
\vspace{1cm}



\begin{columns}[t]
\begin{column}{0.5\textwidth}
 \centering


\centering
 \textbf{\large Principio de coherencia}

(regla del producto)

\begin{equation*}
 P(\text{Hipótesis}|\text{Dato})  = \frac{P(\text{Hipótesis},\text{Dato})}{P(\text{Dato})}
\end{equation*}

\vspace{0.1cm}

\small
Preservamos la creencia previa que \\
sigue siendo compatible con el dato


 \end{column}
 \begin{column}{0.5\textwidth}
 \centering

 \textbf{\large Principio de integridad}

(regla de la suma)

\begin{equation*}
 P(\text{Dato}) = \sum_{\text{\tiny Hipótesis}} P(\text{Hipótesis},\text{Dato})
\end{equation*}

\vspace{0.15cm}

 \small
Predecimos con la contribución de todas las \\
hipótesis. No perdemos ni creamos creencia.

\end{column}
\end{columns}
}
\end{frame}
}
%
% \begin{frame}[plain]
% \begin{textblock}{160}(0,4)
% \centering \LARGE Regla del producto
% \end{textblock}
%
% \begin{textblock}{180}(-10,28)
%  \begin{align*}
%  \phantom{\frac{P(\text{R})}{P\text{R})}P(\text{Regalo} = i)}P(\,\overbrace{\,\text{Regalo} = i\,}^{\text{\small Hipótesis}_i}\,|\,\overbrace{\,\text{Pista} = 2\,}^{\text{\small Dato}}\,) & = \frac{P(\text{Regalo} = i, \text{Pista} = 2)}{P(\text{Pista} = 2)} \phantom{-----------} \\[0.4cm]
%  \only<2-3>{\phantom{\frac{P(\text{R})}{P\text{R})}}P(\,\,\text{Pista} = 2\,\,|\,\,\text{Regalo} = i\,\,) & = }  \only<3>{\frac{P(\text{Regalo} = i, \text{Pista} = 2)}{P(\text{Regalo} = i)}\phantom{\frac{P(\text{R})}{P\text{R})}}}
%  \only<4>{\phantom{\frac{P(\text{R})}{P\text{R})}}P(\text{Regalo} = i)P(\,\,\text{Pista} = 2\,\,|\,\,\text{Regalo} = i\,\,) & =   {P(\text{Regalo} = i, \text{Pista} = 2)}\phantom{\frac{P(\text{R})}{P\text{R})}} }
%  \only<5>{\phantom{\frac{P(\text{R})}{P\text{R})}}P(\text{Regalo} = i)P(\,\,\text{Pista} = 2\,\,|\,\,\text{Regalo} = i\,\,) & =   \bm{P(\text{Regalo} = i, \text{Pista} = 2)}\phantom{\frac{P(\text{R})}{P\text{R})}} }
%  \end{align*}
% \end{textblock}
%
% \end{frame}


\begin{frame}[plain]
\begin{textblock}{160}(0,4)
\centering \LARGE Regla del producto \\
\Large Teorema de Bayes \\
\end{textblock}

\only<1>{
\begin{textblock}{160}(0,26)
\begin{equation*}
\overbrace{P(\text{Hip\'otesis}_i,\,\text{Datos})}^{\text{\small Creencia que sobrevive}} = \overbrace{P(\text{Datos}\,|\,\text{Hip\'otesis}_i)}^{\text{\small Predicción (verosimilitud) }} \overbrace{P(\text{Hip\'otesis}_i)}^{\text{\small Creencia previa}}
\end{equation*}
\end{textblock}
}

\only<2>{
\begin{textblock}{130}(30,23)
\begin{flalign*}
& \underbrace{P(\text{Hip\'otesis}_i|\,\text{Datos})}_{\text{\small Posterior}} = \frac{\overbrace{P(\text{Hip\'otesis}_i,\, \text{Datos})}^{\text{\small Creencia que sobrevive}} }{\underbrace{P(\text{Datos})}_{\text{\small Evidencia}}} &&
\end{flalign*}
\end{textblock}
}


\only<3>{
\begin{textblock}{130}(30,23)
\begin{flalign*}
& \underbrace{P(\text{Hip\'otesis}_i|\,\text{Datos})}_{\text{\small Posterior}} = \frac{\overbrace{P(\text{Dato}\,|\,\text{Hip\'otesis}_i)}^{\text{\small Verosimilitud}} \overbrace{P(\text{Hip\'otesis}_i)}^{\text{\small Prior}} }{\underbrace{P(\text{Dato})}_{\text{\small Evidencia}}} &&
\end{flalign*}
\end{textblock}
}

\vspace{0.2cm}

\only<4->{
%\vspace{0.3cm}
\Wider[2cm]{
\begin{textblock}{160}(0,23)
\begin{equation*}
\underbrace{P(\text{Hip\'otesis}_i|\,\text{Datos, Modelo})}_{\text{\small Posterior}} = \frac{\overbrace{P(\text{Datos}\,|\,\text{Hip\'otesis$_i$, Modelo})}^{\text{\small Verosimilitud}} \overbrace{P(\text{Hip\'otesis}_i|\text{ Modelo})}^{\text{\small Prior}} }{\underbrace{P(\text{Datos }|\text{ Modelo})}_{\text{\small Evidencia}}}
\end{equation*}
\end{textblock}
}
}

\only<4>{
\begin{textblock}{100}(30,65)
\centering \vspace{0.05cm}
\Large Y el \textbf{modelo}!

\large que relaciona el \textbf{dato} con la \textbf{hipótesis}!
\vspace{0.1cm}
\end{textblock}
}


\only<5-7>{
\begin{textblock}{140}(10,58)
\begin{flalign*}
\only<5>{\overbrace{P(\text{Datos}\,|\,\text{Modelo})}^{\text{\small Evidencia}} = \ ? }
\only<6>{\overbrace{P(\text{Datos}\,|\,\text{Modelo})}^{\text{\small Evidencia}} = \ \text{\Large \ Regla de la suma} }
\only<7>{\overbrace{P(\text{Datos}\,|\,\text{Modelo} )}^{\text{\small Evidencia}} = }
\only<7>{\sum_{\text{Hipótesis}_i} P(\underbrace{\text{Hip\'otesis$_i$},\,\text{Dato}}_{\hfrac{\text{\small Creencia}}
{\text{\small conjunta}}} \, | \, \text{Modelo})}
&&
\end{flalign*}
\end{textblock}
}


\only<8->{
\begin{textblock}{160}(0,60)
\begin{equation*}
 P(\text{Modelo}_j|\text{Datos}) = \onslide<9>{\frac{\overbrace{P(\text{Datos}|\text{Modelo}_j)}^{\text{\small Evidencia}} P(\text{Modelo}_j)}{ P(\text{Datos})}}
\end{equation*}
\end{textblock}
}


\end{frame}



\begin{frame}[plain]
\begin{textblock}{160}(0,4)
 \centering \LARGE Modelo causal alternativo
 \end{textblock}
 \vspace{-1cm}

 \begin{textblock}{80}(0,24)
 \centering

 \large Modelo gráfico:

 \vspace{0.3cm}

 \tikz{
    \only<-2>{\phantom}{\node[latent] (d) {\includegraphics[width=0.10\textwidth]{../../auxiliar/static/dedo.png}} ;}
    \only<-2>{\phantom}{\node[const,above=of d] (nd) {\Large $s$} ;}
    \node[latent, above=of d, xshift=-1.5cm] (r) {\includegraphics[width=0.12\textwidth]{../../auxiliar/static/regalo.png}} ;
    \node[const,below=of r] (nr) {\Large $r$} ;
    \only<-1>{\phantom}{\node[latent, fill=black!30, above=of d, xshift=1.5cm] (c) {\includegraphics[width=0.12\textwidth]{../../auxiliar/static/cerradura.png}} ;}
    \only<-1>{\phantom}{\node[const,below=of c] (nc) {\, \Large $c = 1$} ;}
    \only<-2>{\phantom}{\edge {r,c} {d};}

    \only<-2>{\phantom}{\node[const,below=of d] (modelo) {\large $s \neq r$ \, $s \neq c$} ;}
}
 \end{textblock}


\only<1>{
 \begin{textblock}{160}(80,33)
\scalebox{1.5}{
\tikz{
    \node[factor, minimum size=1cm] (p1) {} ;
    \node[factor, minimum size=1cm, xshift=1.5cm] (p2) {} ;
    \node[factor, minimum size=1cm, xshift=3cm] (p3) {} ;

    \node[const, above=of p1, yshift=.15cm] (fp1) {$1/3$};
    \node[const, above=of p2, yshift=.15cm] (fp2) {$1/3$};
    \node[const, above=of p3, yshift=.15cm] (fp3) {$1/3$};
    \node[const, below=of p2, yshift=-.10cm, xshift=0.3cm] (dedo) {};

    \node[invisible, xshift=4.75cm] (s-dist) {};
    \node[invisible, yshift=-1cm] (s-dist) {};
    \node[invisible, yshift=1.2cm] (s-dist) {};
    }
}
\end{textblock}
}

\only<2>{
 \begin{textblock}{160}(80,33)
\scalebox{1.5}{
\tikz{
    \node[factor, minimum size=1cm] (p1) {\includegraphics[width=0.025\textwidth]{../../auxiliar/static/cerradura.png}} ;
    \node[factor, minimum size=1cm, xshift=1.5cm] (p2) {} ;
    \node[factor, minimum size=1cm, xshift=3cm] (p3) {} ;

    \node[const, above=of p1, yshift=.15cm] (fp1) {$1/3$};
    \node[const, above=of p2, yshift=.15cm] (fp2) {$1/3$};
    \node[const, above=of p3, yshift=.15cm] (fp3) {$1/3$};
    \node[const, below=of p2, yshift=-.10cm, xshift=0.3cm] (dedo) {};

    \node[invisible, xshift=4.75cm] (s-dist) {};
    \node[invisible, yshift=-1cm] (s-dist) {};
    \node[invisible, yshift=1.2cm] (s-dist) {};
    }
}
\end{textblock}
}


\only<3>{
 \begin{textblock}{160}(80,33)
\scalebox{1.5}{
\tikz{
    \node[factor, minimum size=1cm] (p1) {\includegraphics[width=0.025\textwidth]{../../auxiliar/static/cerradura.png}} ;
    \node[det, minimum size=1cm, xshift=1.5cm] (p2) {\includegraphics[width=0.03\textwidth]{../../auxiliar/static/dedo.png}} ;
    \node[factor, minimum size=1cm, xshift=3cm] (p3) {} ;

    \node[const, above=of p1, yshift=.15cm] (fp1) {$\phantom{/}?\phantom{/}$};
    \node[const, above=of p2, yshift=.15cm] (fp2) {$\phantom{/}0\phantom{/}$};
    \node[const, above=of p3, yshift=.15cm] (fp3) {$\phantom{/}?\phantom{/}$};
    \node[const, below=of p2, yshift=-.10cm, xshift=0.3cm] (dedo) {};

    \node[invisible, xshift=4.75cm] (s-dist) {};
    \node[invisible, yshift=-1cm] (s-dist) {};
    \node[invisible, yshift=1.2cm] (s-dist) {};
    }
}
\end{textblock}
}

\end{frame}


\begin{frame}[plain]
\begin{textblock}{160}(0,4)
 \centering \LARGE Modelo causal alternativo \\
 \Large \phantom{y el dato} Incertidumbre óptima dado el modelo \only<1-12>{\phantom}{y el dato}
 \end{textblock}
 \vspace{-1cm}

 \only<1-3>{
 \begin{textblock}{80}(0,24)
 \centering

 \large Modelo gráfico:

 \vspace{0.3cm}

 \tikz{
    {\node[latent] (d) {\includegraphics[width=0.10\textwidth]{../../auxiliar/static/dedo.png}} ;}
    {\node[const,above=of d] (nd) {\Large $s$} ;}
    \node[latent, above=of d, xshift=-1.5cm] (r) {\includegraphics[width=0.12\textwidth]{../../auxiliar/static/regalo.png}} ;
    \node[const,below=of r] (nr) {\Large $r$} ;
    {\node[latent, fill=black!30, above=of d, xshift=1.5cm] (c) {\includegraphics[width=0.12\textwidth]{../../auxiliar/static/cerradura.png}} ;}
    {\node[const,below=of c] (nc) {\, \Large $c = 1$} ;}
    {\edge {r,c} {d};}

    {\node[const,below=of d] (modelo) {\large $s \neq r$ \, $s \neq c$} ;}
}
 \end{textblock}
}

  \only<4-12>{
 \begin{textblock}{80}(0,26)
  \centering
  $P(r,s)$ \\ \vspace{0.3cm}
 \begin{tabular}{c|c|c|c||c} \setlength\tabcolsep{0.4cm}
        & \, $r_1$ \, &  \, $r_2$ \, & \, $r_3$ \, & \\ \hline
  { $s_2$}  & \onslide<5->{$1/6$} & \onslide<7->{$0$} & \onslide<9->{$1/3$} & \onslide<12->{$1/2$} \\ \hline
       {$s_3$} & \onslide<6->{$1/6$} & \onslide<8->{$1/3$} & \onslide<10->{$0$} & \onslide<12->{$1/2$} \\ \hline
              & \onslide<12->{$1/3$} &  \onslide<12->{$1/3$} & \onslide<12->{$1/3$}  & \onslide<12->{$1$} \\
\end{tabular}
\end{textblock}
}

\only<13>{
 \begin{textblock}{80}(0,26)
  \centering
  $P(r,s_2)$ \\ \vspace{0.3cm}
 \begin{tabular}{c|c|c|c||c} \setlength\tabcolsep{0.4cm}
        & \, $r_1$ \, &  \, $r_2$ \, & \, $r_3$ \, & \\ \hline
        { $s_2$}  & \onslide<6->{$1/6$} & \onslide<8->{$0$} & \onslide<10->{$1/3$} & \onslide<13->{$1/2$} \\ \hline
\end{tabular}
\end{textblock}
}


\only<14->{
 \begin{textblock}{80}(0,26)
  \centering
  $P(r|s_2)$ \\ \vspace{0.3cm}
 \begin{tabular}{c|c|c|c||c} \setlength\tabcolsep{0.4cm}
        & \, $r_1$ \, &  \, $r_2$ \, & \, $r_3$ \, & \phantom{$1/2$}\\ \hline
  { $s_2$}  & \onslide<6->{$1/3$} & \onslide<8->{$0$} & \onslide<10->{$2/3$} & \onslide<13->{$1$} \\ \hline
\end{tabular}
\end{textblock}
}


\only<11-12>{
\begin{textblock}{80}(0,58)
 \centering
\begin{center}
 Regla de la suma
 \end{center}

 $P(s_i) = \sum_{j} P(r_j,s_i)$
 \\

\end{textblock}
}

\only<13-14>{
\begin{textblock}{80}(0,58)
 \centering
\begin{center}
 Regla del producto
 \end{center}
 \begin{equation*}
P(r_i|s_2) = \frac{P(r_i,s_2)}{P(s_2)}
 \end{equation*}

\end{textblock}
}


\only<15>{
\begin{textblock}{70}(10,55)
\centering
 \scalebox{1}{
\tikz{
    \node[factor, minimum size=1cm] (p1) {\includegraphics[width=0.07\textwidth]{../../auxiliar/static/cerradura.png}} ;
    \node[det, minimum size=1cm, xshift=1.5cm] (p2) {\includegraphics[width=0.07\textwidth]{../../auxiliar/static/dedo.png}} ;
    \node[factor, minimum size=1cm, xshift=3cm] (p3) {} ;

    \node[const, above=of p1, yshift=.15cm] (fp1) {$1/3$};
    \node[const, above=of p2, yshift=.15cm] (fp2) {$\phantom{/}0\phantom{/}$};
    \node[const, above=of p3, yshift=.15cm] (fp3) {$2/3$};
    \node[const, below=of p2, yshift=-.10cm, xshift=0.3cm] (dedo) {};

    \node[invisible, xshift=4.75cm] (s-dist) {};
    \node[invisible, yshift=-1cm] (s-dist) {};
    \node[invisible, yshift=1.2cm] (s-dist) {};
    }
}
\end{textblock}
}

 \only<2-12>{
\begin{textblock}{80}(70,20) \centering
\scalebox{1.2}{
 \tikz{
 \onslide<2->{
\node[latent, draw=white, yshift=0.8cm] (b0) {$1$};
\node[latent,below=of b0,yshift=0.8cm, xshift=-2cm] (r1) {$r_1$};
{\node[latent,below=of b0,yshift=0.8cm] (r2) {$r_2$}; }
\node[latent,below=of b0,yshift=0.8cm, xshift=2cm] (r3) {$r_3$};
\node[latent, below=of r1, draw=white, yshift=0.7cm] (bc11) {$\frac{1}{3}$};
{\node[latent, below=of r2, draw=white, yshift=0.7cm] (bc12) {$\frac{1}{3}$};}
\node[latent, below=of r3, draw=white, yshift=0.7cm] (bc13) {$\frac{1}{3}$};
}
\onslide<3->{
\node[latent,below=of bc11,yshift=0.7cm, xshift=-0.5cm] (r1d2) {$s_2$};
{\node[latent,below=of bc11,yshift=0.7cm, xshift=0.5cm] (r1d3) {$s_3$};}
{\node[latent,below=of bc12,yshift=0.7cm] (r2d3) {$s_3$};}
\node[latent,below=of bc13,yshift=0.7cm] (r3d2) {$s_2$};
\node[latent,below=of r1d2,yshift=0.7cm,draw=white] (br1d2) {$\only<5>{\bm}{\frac{1}{3}\frac{1}{2}}$};
{\node[latent,below=of r1d3,yshift=0.7cm, draw=white] (br1d3) {$\only<6>{\bm}{\frac{1}{3}\frac{1}{2}}$};}
{\node[latent,below=of r2d3,yshift=0.7cm,draw=white] (br2d3) {$\only<8>{\bm}{\frac{1}{3}}$};}
\node[latent,below=of r3d2,yshift=0.7cm,draw=white] (br3d2) {$\only<9>{\bm}{\frac{1}{3}}$};
}

\node[invisible, left=of r1d2,xshift=-0.1cm] (il) {};
\node[invisible, right=of br3d2,xshift=0.1cm] (il) {};

\onslide<2->{
\edge[-] {b0} {r1,r2,r3};
\edge[-] {r1} {bc11};
\edge[-] {r2} {bc12};
\edge[-] {r3} {bc13};
}
\onslide<3->{
\edge[-] {bc11} {r1d2,r1d3};
\edge[-] {bc12} {r2d3};
\edge[-] {bc13} {r3d2};
\edge[-] {r1d2} {br1d2};
\edge[-] {r1d3} {br1d3};
\edge[-] {r2d3} {br2d3};
\edge[-] {r3d2} {br3d2};
}
}
}
\end{textblock}
}


\only<13->{
\begin{textblock}{80}(70,20) \centering
\scalebox{1.2}{
 \tikz{
\node[latent, draw=white, yshift=0.8cm] (b0) {$1$};
\node[latent,below=of b0,yshift=0.8cm, xshift=-2cm] (r1) {$r_1$};
{\color{gray}\node[latent,draw=gray,below=of b0,yshift=0.8cm] (r2) {$r_2$}; }
\node[latent,below=of b0,yshift=0.8cm, xshift=2cm] (r3) {$r_3$};

% \node[latent, below=of r1, draw=white, yshift=0.8cm] (br1) {$\frac{1}{3}$};
% \node[latent, below=of r2, draw=white, yshift=0.8cm] (br2) {$\frac{1}{3}$};
% \node[latent, below=of r3, draw=white, yshift=0.8cm] (br3) {$\frac{1}{3}$};
% \node[latent,below=of br1,yshift=0.8cm] (c11) {$c_1$};
% \node[latent,below=of br2,yshift=0.8cm] (c12) {$c_1$};
% \node[latent,below=of br3,yshift=0.8cm] (c13) {$c_1$};

\node[latent, below=of r1, draw=white, yshift=0.7cm] (bc11) {$\frac{1}{3}$};
{\color{gray}\node[latent, below=of r2, draw=white, yshift=0.7cm] (bc12) {$\frac{1}{3}$};}
\node[latent, below=of r3, draw=white, yshift=0.7cm] (bc13) {$\frac{1}{3}$};
\node[latent,below=of bc11,yshift=0.7cm, xshift=-0.5cm] (r1d2) {$s_2$};
{\color{gray}\node[latent,draw=gray,below=of bc11,yshift=0.7cm, xshift=0.5cm] (r1d3) {$s_3$};}
{\color{gray}\node[latent, draw=gray,below=of bc12,yshift=0.7cm] (r2d3) {$s_3$};}
\node[latent,below=of bc13,yshift=0.7cm] (r3d2) {$s_2$};

\node[latent,below=of r1d2,yshift=0.7cm,draw=white] (br1d2) {$\frac{1}{3}\frac{1}{2}$};
{\color{gray}\node[latent,below=of r1d3,yshift=0.7cm, draw=white] (br1d3) {$\frac{1}{3}\frac{1}{2}$};}
{\color{gray}\node[latent,below=of r2d3,yshift=0.7cm,draw=white] (br2d3) {$\frac{1}{3}$};}
\node[latent,below=of r3d2,yshift=0.7cm,draw=white] (br3d2) {$\frac{1}{3}$};
\edge[-] {b0} {r1,r3};
\edge[-,draw=gray] {b0} {r2};
% \edge[-] {r1} {br1};
% \edge[-] {r2} {br2};
% \edge[-] {r3} {br3};
% \edge[-] {br1} {c11};
% \edge[-] {br2} {c12};
% \edge[-] {br3} {c13};
\edge[-] {r1} {bc11};
\edge[-,draw=gray] {r2} {bc12};
\edge[-] {r3} {bc13};
\edge[-] {bc11} {r1d2};
\edge[-,draw=gray] {bc11} {r1d3};
\edge[-,draw=gray] {bc12} {r2d3};
\edge[-] {bc13} {r3d2};
\edge[-] {r1d2} {br1d2};
\edge[-,draw=gray] {r1d3} {br1d3};
\edge[-,draw=gray] {r2d3} {br2d3};
\edge[-] {r3d2} {br3d2};
}
}
\end{textblock}
}

\end{frame}


\begin{frame}[plain]
\begin{textblock}{160}(0,4)
\centering \LARGE Modelos causales alternativos
\end{textblock}
 \vspace{1.25cm}

\begin{textblock}{80}(80,16)
\centering
 \tikz{
    \node[latent,] (r) {\includegraphics[width=0.12\textwidth]{../../auxiliar/static/regalo.png}} ;
    \node[const,left=of r] (nr) {\Large $r$} ;


    \node[latent, below=of r] (d) {\includegraphics[width=0.10\textwidth]{../../auxiliar/static/dedo.png}} ;
    \node[const, left=of d] (nd) {\Large $s$} ;
    \node[const, below=of d, yshift=-0.2cm] (restricciones) {$s \neq r$};

    \edge {r} {d};

}

\vspace{0.5cm}
\onslide<-1>{
\tikz{
         \node[factor, minimum size=1cm] (p1) {\includegraphics[width=0.07\textwidth]{../../auxiliar/static/cerradura.png}} ;
         \node[det, minimum size=1cm, xshift=1.5cm] (p2) {\includegraphics[width=0.07\textwidth]{../../auxiliar/static/dedo.png}} ;
         \node[factor, minimum size=1cm, xshift=3cm] (p3) {} ;

         \node[const, above=of p1, yshift=.1cm] (fp1) {$1/2$};
         \node[const, above=of p2, yshift=.1cm] (fp2) {$\phantom{/}0\phantom{/}$};
         \node[const, above=of p3, yshift=.1cm] (fp3) {$1/2$};
         \node[const, below=of p2, yshift=-.10cm, xshift=0.3cm] (dedo) {};

        }
}

\end{textblock}



\begin{textblock}{80}(0,16)
\centering
\tikz{

    \node[latent] (d) {\includegraphics[width=0.10\textwidth]{../../auxiliar/static/dedo.png}} ;
    \node[const,left=of d] (nd) {\Large $s$} ;
    \node[const, below=of d, yshift=-0.2cm] (restricciones) {$s \neq r \text{, } s \neq c$};


    \node[latent, above=of d, xshift=-1.5cm] (r) {\includegraphics[width=0.12\textwidth]{../../auxiliar/static/regalo.png}} ;
    \node[const,left=of r] (nr) {\Large $r$} ;


    \node[latent, fill=black!30, above=of d, xshift=1.5cm] (c) {\includegraphics[width=0.12\textwidth]{../../auxiliar/static/cerradura.png}} ;
    \node[const,left=of c] (nc) {\Large $c$} ;

    \edge {r,c} {d};
}

\vspace{0.5cm}
\onslide<-1>{
\tikz{
         \node[factor, minimum size=1cm] (p1) {\includegraphics[width=0.07\textwidth]{../../auxiliar/static/cerradura.png}} ;
         \node[det, minimum size=1cm, xshift=1.5cm] (p2) {\includegraphics[width=0.07\textwidth]{../../auxiliar/static/dedo.png}} ;
         \node[factor, minimum size=1cm, xshift=3cm] (p3) {} ;

         \node[const, above=of p1, yshift=.1cm] (fp1) {$1/3$};
         \node[const, above=of p2, yshift=.1cm] (fp2) {$\phantom{/}0\phantom{/}$};
         \node[const, above=of p3, yshift=.1cm] (fp3) {$2/3$};
         \node[const, below=of p2, yshift=-.10cm, xshift=0.3cm] (dedo) {};

        }
}

\end{textblock}


\only<2>{
\begin{textblock}{160}(0,70)
\centering \Large ¿Y el acuerdo intersubjetivo respecto de los modelos?  \\[0.2cm] \large
$P(\text{Modelo}_i|\text{Datos})$

\end{textblock}
}

\end{frame}



\begin{frame}[plain,fragile]
\begin{textblock}{160}(0,4)
\centering \LARGE Evaluación de modelos causales
\end{textblock}
\vspace{1cm}

\only<-5>{
\begin{textblock}{160}(0,18)
\begin{equation*}
 P(\text{Modelo}_i|\text{Datos}) = \frac{\overbrace{P(\text{Datos}|\text{Modelo}_i)}^{\text{\small Evidencia} } P(\text{Modelo}_i)}{ P(\text{Datos})}
\end{equation*}
\end{textblock}
}
%
% \only<2>{
% \begin{textblock}{160}(0,44)
% \begin{equation*}
% P(\text{Hip\'otesis}_i|\,\text{Datos, Modelo}) = \frac{P(\text{Datos}\,|\,\text{Hip\'otesis$_i$, Modelo}) P(\text{Hip\'otesis}_i|\text{ Modelo})} {\underbrace{P(\text{Datos }|\text{ Modelo})}_{\hfrac{\text{\footnotesize Predicción a priori}}{\text{\footnotesize o evidencia}}} }
% \end{equation*}
% \end{textblock}
% }
%

%
% \only<3-4>{
% \begin{textblock}{160}(0,42)
%  \begin{equation*}
% \begin{split}
%  \frac{P(\text{Modelo}_A|\text{Datos})}{P(\text{Modelo}_B|\text{Datos})} = \frac{P(\text{Datos}|\text{Modelo}_A)} {P(\text{Datos}|\text{Modelo}_B)} \only<4>{\phantom}{\frac{P(\text{Modelo}_A)}{P(\text{Modelo}_B)}}
% \end{split}
% \end{equation*}
% \end{textblock}
% }

\only<2->{
\begin{textblock}{130}(18,38)
 \begin{flalign*}
 P(\text{Dat\en{a}\es{os}} = \{d_1, \, d_2, \, \dots \} | \, \text{Model\es{o}}) & =
\only<3>{ \text{\Large \,Regla de la suma}}
\only<4>{ \text{\Large \,Regla del producto}}
 \only<5->{P(d_1|\text{Model\es{o}})P(d_2|d_1,\text{Model\es{o}}) \dots} &&
\end{flalign*}
\end{textblock}
}


\only<6,10>{
\begin{textblock}{80}(60,22)
\tikz{
    \node[factor, minimum size=1cm] (p1) {\includegraphics[width=0.07\textwidth]{../../auxiliar/static/cerradura.png}} ;
    \node[factor, minimum size=1cm, xshift=1.5cm] (p2) {} ;
    \node[factor, minimum size=1cm, xshift=3cm] (p3) {} ;
}
\end{textblock}
}
\only<7-8>{
\begin{textblock}{80}(60,22)
\tikz{
    \node[factor, minimum size=1cm] (p1) {\includegraphics[width=0.07\textwidth]{../../auxiliar/static/cerradura.png}} ;
    \node[det, minimum size=1cm, xshift=1.5cm] (p2) {\includegraphics[width=0.07\textwidth]{../../auxiliar/static/dedo.png}} ;
    \node[factor, minimum size=1cm, xshift=3cm] (p3) {} ;
}
\end{textblock}
}
\only<9>{
\begin{textblock}{80}(60,22)
\tikz{
    \node[det, minimum size=1cm] (p1) {\includegraphics[width=0.07\textwidth]{../../auxiliar/static/regalo.png}} ;
    \node[det, minimum size=1cm, xshift=1.5cm] (p2) {} ;
    \node[det, minimum size=1cm, xshift=3cm] (p3) {} ;
}
\end{textblock}
}
\only<11-12>{
\begin{textblock}{80}(60,22)
\tikz{
    \node[factor, minimum size=1cm] (p1) {\includegraphics[width=0.07\textwidth]{../../auxiliar/static/cerradura.png}} ;
    \node[factor, minimum size=1cm, xshift=1.5cm] (p2) {} ;
    \node[det, minimum size=1cm, xshift=3cm] (p3) {\includegraphics[width=0.07\textwidth]{../../auxiliar/static/dedo.png}} ;
}
\end{textblock}
}
\only<13>{
\begin{textblock}{80}(60,22)
\tikz{
    \node[det, minimum size=1cm] (p1) {} ;
    \node[det, minimum size=1cm, xshift=1.5cm] (p2) {\includegraphics[width=0.07\textwidth]{../../auxiliar/static/regalo.png}} ;
    \node[det, minimum size=1cm, xshift=3cm] (p3) {} ;
}
\end{textblock}
}


\only<6>{
\begin{textblock}{80}(0,52) \centering
\begin{tabular}{|c|c|c|c||c|} \hline  \setlength\tabcolsep{0.4cm}
\phantom{$\bm{s_2}$} & \, $r_1$ \, &  \, $r_2$ \, & \, $r_3$ \, & \phantom{$\bm{1/2}$} \\ \hline
  $s_1$ & $0$ & $0$ & $0$ &   $0$ \\ \hline
  $s_2$ & $1/6$ & $0$ & $1/3$ &  $1/2$ \\  \hline
  $s_3$ & $1/6$ & $1/3$ & $0$ & $1/2$ \\ \hline
  \end{tabular}
\end{textblock}
}
\only<7>{
\begin{textblock}{80}(0,52) \centering
\begin{tabular}{|c|c|c|c||c|} \hline  \setlength\tabcolsep{0.4cm}
\phantom{$\bm{s_2}$} & \, $r_1$ \, &  \, $r_2$ \, & \, $r_3$ \, &  \phantom{$\bm{1/2}$}  \\ \hline
  $\gray s_1$ & $\gray0$ & $\gray0$ & $\gray0$ &   $\gray 0$ \\ \hline
  $\bm{s_2}$ & $1/6$ & $0$ & $1/3$ &  $\bm{1/2}$ \\  \hline
  $\gray s_3$ & $\gray1/6$ & $\gray1/3$ & $\gray0$ & $\gray1/2$ \\ \hline
  \end{tabular}
\end{textblock}
}
\only<8>{
\begin{textblock}{80}(0,52) \centering
\begin{tabular}{|c|c|c|c||c|} \hline  \setlength\tabcolsep{0.4cm}
\phantom{$\bm{s_2}$} & \, $r_1$ \, &  \, $r_2$ \, & \, $r_3$ \, & \phantom{$\bm{1/2}$}  \\ \hline
            & & &  &  \\ \hline
  $s_2$ & $1/3$ & $0$ & $2/3$ &  1 \\  \hline
 & & & &\\ \hline
  \end{tabular}
\end{textblock}
}
\only<9>{
\begin{textblock}{80}(0,52) \centering
\begin{tabular}{|c|c|c|c||c|} \hline  \setlength\tabcolsep{0.4cm}
\phantom{$\bm{s_2}$} & \, $\bm{r_1}$ \, &  \, $r_2$ \, & \, $r_3$ \, & \phantom{$\bm{1/2}$}  \\ \hline
            & & &  &  \\ \hline
  $s_2$ & $\bm{1/3}$ & $0$ & $2/3$ &  1 \\  \hline
 & & & &\\ \hline
  \end{tabular}
\end{textblock}
}
\only<10>{
\begin{textblock}{80}(0,52) \centering
\begin{tabular}{|c|c|c|c||c|} \hline  \setlength\tabcolsep{0.4cm}
\phantom{$\bm{s_2}$} & \, $r_1$ \, &  \, $r_2$ \, & \, $r_3$ \, & \phantom{$\bm{1/2}$} \\ \hline
  $s_1$ & $0$ & $0$ & $0$ &   $0$ \\ \hline
  $s_2$ & $1/6$ & $0$ & $1/3$ &  $1/2$ \\  \hline
  $s_3$ & $1/6$ & $1/3$ & $0$ & $1/2$ \\ \hline
  \end{tabular}
\end{textblock}
}
\only<11>{
\begin{textblock}{80}(0,52) \centering
\begin{tabular}{|c|c|c|c||c|} \hline  \setlength\tabcolsep{0.4cm}
\phantom{$\bm{s_2}$} & \, $r_1$ \, &  \, $r_2$ \, & \, $r_3$ \, &  \phantom{$\bm{1/2}$}  \\ \hline
  $\gray s_1$ & $\gray0$ & $\gray0$ & $\gray0$ &   $\gray 0$ \\ \hline
  $\gray s_2$ & $\gray1/6$ & $\gray0$ & $\gray1/3$ &  $\gray1/2$ \\  \hline
  $\bm{s_3}$ & $1/6$ & $1/3$ & $0$ & $\bm{1/2}$ \\ \hline
  \end{tabular}
\end{textblock}
}
\only<12>{
\begin{textblock}{80}(0,52) \centering
\begin{tabular}{|c|c|c|c||c|} \hline  \setlength\tabcolsep{0.4cm}
\phantom{$\bm{s_2}$} & \, $r_1$ \, &  \, $r_2$ \, & \, $r_3$ \, & \phantom{$\bm{1/2}$}  \\ \hline
            & & &  &  \\ \hline
 & & & &\\ \hline
 $s_3$ & $1/3$ & $2/3$ & $0$ &  1 \\  \hline
  \end{tabular}
\end{textblock}
}
\only<13>{
\begin{textblock}{80}(0,52) \centering
\begin{tabular}{|c|c|c|c||c|} \hline  \setlength\tabcolsep{0.4cm}
\phantom{$\bm{s_2}$} & \, $r_1$ \, &  \, $\bm{r_2}$ \, & \, $r_3$ \, & \phantom{$\bm{1/2}$}  \\ \hline
            & & &  &  \\ \hline
 & & & &\\ \hline
 $s_3$ & $1/3$ & $\bm{2/3}$ & $0$ &  1 \\  \hline
  \end{tabular}
\end{textblock}
}
\only<6>{
\begin{textblock}{80}(80,52) \centering
\begin{tabular}{|c|c|c|c||c|} \hline  \setlength\tabcolsep{0.4cm}
\phantom{$\bm{s_2}$} & \, $r_1$ \, &  \, $r_2$ \, & \, $r_3$ \, & \phantom{$\bm{1/3}$}  \\ \hline
  $s_1$ & $0$ & $1/6$ & $1/6$ &   $1/3$ \\ \hline
  $s_2$ & $1/6$ & $0$ & $1/6$ &  $1/3$ \\  \hline
  $s_3$ & $1/6$ & $1/6$ & $0$ & $1/3$ \\ \hline
  \end{tabular}
\end{textblock}
}
\only<7>{
\begin{textblock}{80}(80,52) \centering
\begin{tabular}{|c|c|c|c||c|} \hline  \setlength\tabcolsep{0.4cm}
 \phantom{$\bm{s_2}$} & \, $r_1$ \, &  \, $r_2$ \, & \, $r_3$ \, & \phantom{$\bm{1/3}$} \\ \hline
  $\gray s_1$ & $\gray0$ & $\gray1/6$ & $\gray1/6$ &   $\gray 1/3$ \\ \hline
  $\bm{s_2}$ & $1/6$ & $0$ & $1/6$ &  $\bm{1/3}$ \\  \hline
  $\gray s_3$ & $\gray1/6$ & $\gray1/6$ & $\gray0$ & $\gray1/3$ \\ \hline
  \end{tabular}
\end{textblock}
}
\only<8>{
\begin{textblock}{80}(80,52) \centering
\begin{tabular}{|c|c|c|c||c|} \hline  \setlength\tabcolsep{0.4cm}
\phantom{$\bm{s_2}$} & \, $r_1$ \, &  \, $r_2$ \, & \, $r_3$ \, & \phantom{$\bm{1/2}$}  \\ \hline
            & & &  &  \\ \hline
  $s_2$ & $1/2$ & $0$ & $1/2$ &  1 \\  \hline
 & & & &\\ \hline
  \end{tabular}
\end{textblock}
}
\only<9>{
\begin{textblock}{80}(80,52) \centering
\begin{tabular}{|c|c|c|c||c|} \hline  \setlength\tabcolsep{0.4cm}
\phantom{$\bm{s_2}$} & \, $\bm{r_1}$ \, &  \, $r_2$ \, & \, $r_3$ \, & \phantom{$\bm{1/2}$}  \\ \hline
            & & &  &  \\ \hline
  $s_2$ & $\bm{1/2}$ & $0$ & $1/2$ &  1 \\  \hline
 & & & &\\ \hline
  \end{tabular}
\end{textblock}
}
\only<10>{
\begin{textblock}{80}(80,52) \centering
\begin{tabular}{|c|c|c|c||c|} \hline  \setlength\tabcolsep{0.4cm}
\phantom{$\bm{s_2}$} & \, $r_1$ \, &  \, $r_2$ \, & \, $r_3$ \, & \phantom{$\bm{1/3}$}  \\ \hline
  $s_1$ & $0$ & $1/6$ & $1/6$ &   $1/3$ \\ \hline
  $s_2$ & $1/6$ & $0$ & $1/6$ &  $1/3$ \\  \hline
  $s_3$ & $1/6$ & $1/6$ & $0$ & $1/3$ \\ \hline
  \end{tabular}
\end{textblock}
}
\only<11>{
\begin{textblock}{80}(80,52) \centering
\begin{tabular}{|c|c|c|c||c|} \hline  \setlength\tabcolsep{0.4cm}
 \phantom{$\bm{s_2}$} & \, $r_1$ \, &  \, $r_2$ \, & \, $r_3$ \, & \phantom{$\bm{1/3}$} \\ \hline
  $\gray s_1$ & $\gray0$ & $\gray1/6$ & $\gray1/6$ &   $\gray 1/3$ \\ \hline
  $\gray s_2$ & $\gray1/6$ & $\gray0$ & $\gray1/6$ &  $\gray1/3$ \\  \hline
  $\bm{s_3}$ & $1/6$ & $1/6$ & $0$ & $\bm{1/3}$ \\ \hline
  \end{tabular}
\end{textblock}
}
\only<12>{
\begin{textblock}{80}(80,52) \centering
\begin{tabular}{|c|c|c|c||c|} \hline  \setlength\tabcolsep{0.4cm}
\phantom{$\bm{s_2}$} & \, $r_1$ \, &  \, $r_2$ \, & \, $r_3$ \, & \phantom{$\bm{1/2}$}  \\ \hline
            & & &  &  \\ \hline
 & & & &\\ \hline
 $s_3$ & $1/2$ & $1/2$ & $0$ &  1 \\  \hline
  \end{tabular}
\end{textblock}
}
\only<13>{
\begin{textblock}{80}(80,52) \centering
\begin{tabular}{|c|c|c|c||c|} \hline  \setlength\tabcolsep{0.4cm}
\phantom{$\bm{s_2}$} & \, $r_1$ \, &  \, $\bm{r_2}$ \, & \, $r_3$ \, & \phantom{$\bm{1/2}$}  \\ \hline
            & & &  &  \\ \hline
 & & & &\\ \hline
 $s_3$ & $1/2$ & $\bm{1/2}$ & $0$ &  1 \\  \hline
  \end{tabular}
\end{textblock}
}



\begin{textblock}{80}(0,73) \centering
 \begin{equation*}
 \onslide<6->{ P(\text{D}|\text{M}) }  \onslide<7->{= \frac{1}{2}} \, \onslide<9->{\frac{1}{3}} \, \onslide<11->{\frac{1}{2}} \, \onslide<13>{\frac{2}{3}}
 \end{equation*}
\end{textblock}
\begin{textblock}{80}(80,73) \centering
\begin{equation*}
 \onslide<6->{ P(\text{D}|\text{M})}  \onslide<7->{= \frac{1}{3}} \, \onslide<9->{\frac{1}{2}} \, \onslide<11->{\frac{1}{3}} \, \onslide<13>{\frac{1}{2}}
 \end{equation*}
\end{textblock}



\end{frame}



\begin{frame}[plain]
\begin{textblock}{160}(0,4)
\centering \LARGE Evaluación de modelos causales
%\\ \Large Datos generados con el modelo Monty Hall
\end{textblock}
%
% \begin{textblock}{160}(14,12)
% \begin{equation*}
%  P(\text{Modelo}|\text{Datos}) = \frac{\only<1->{\overbrace{P(\text{Data}|\text{Modelo})}^{\text{\footnotesize Predicción a priori}}} \only<1->{P(\text{Modelo})} }{ P(\text{Data})} \phantom{\frac{\overbrace{P(\text{Datos}|\text{Modelo})}^{\text{Evidencia}}}{ P(\text{Datos})}}
% \end{equation*}
% \end{textblock}
% %
% \only<2>{
% \begin{textblock}{160}(0,47)
% \begin{align*}
% P(\text{Data}|\text{Modelo}) & = \sum_{i} P(\text{Data}|\text{Hypothesis}_i,\text{Model}) P(\text{Hypothesis}_i|\text{Model})
% \end{align*}
% \end{textblock}
% }



\only<1>{

\begin{textblock}{140}(10,26)
\centering
\includegraphics[width=0.66\textwidth]{figuras/monty_hall_selection.pdf} \hspace{2cm}
\end{textblock}

\begin{textblock}{80}(86,26)
\centering
\scalebox{0.5}{
\tikz{

    \node[latent] (d) {\includegraphics[width=0.10\textwidth]{../../auxiliar/static/dedo.png}} ;
    \node[const,left=of d] (nd) {\Large $s$} ;

    \node[latent, above=of d, xshift=-1.5cm] (r) {\includegraphics[width=0.12\textwidth]{../../auxiliar/static/regalo.png}} ;
    \node[const,left=of r] (nr) {\Large $r$} ;


    \node[latent, fill=black!30, above=of d, xshift=1.5cm] (c) {\includegraphics[width=0.12\textwidth]{../../auxiliar/static/cerradura.png}} ;
    \node[const,left=of c] (nc) {\Large $c$} ;

    \edge {r,c} {d};
}
}
\end{textblock}


\begin{textblock}{80}(86,60)
\centering
\scalebox{0.5}{
 \tikz{
    \node[latent,] (r) {\includegraphics[width=0.12\textwidth]{../../auxiliar/static/regalo.png}} ;
    \node[const,left=of r] (nr) {\Large $r$} ;


    \node[latent, below=of r] (d) {\includegraphics[width=0.10\textwidth]{../../auxiliar/static/dedo.png}} ;
    \node[const, left=of d] (nd) {\Large $s$} ;

    \edge {r} {d};

}
}
\end{textblock}
}

\end{frame}

%
%
%
% \begin{frame}[plain]
% \begin{textblock}{160}(0,4)
%  \centering \LARGE Boole \\
%  \large Valor de verdad continuo
%  \end{textblock}
%  \vspace{1.5cm} \centering
%
%  \includegraphics[width=1\textwidth]{../../auxiliar/static/boole.png}
%
%  \normalsize
%  \hfill George Boole (1854) \href{https://downloads.tuxfamily.org/openmathdep/logic_ante_1900/Laws_of_Thought-Boole.pdf}{\emph{An Investigation of the Laws of Thought}}
% \end{frame}

%
% \begin{frame}[plain]
% \begin{textblock}{160}(0,4)
%  \centering \LARGE Lógica de primer orden extendida \\
%  \Large (De una práctica suspendida)
%  \end{textblock}
%  \vspace{1.5cm}
%
% Dado \ \ $A \Rightarrow B  \equiv P(B|A) = 1$
%
% \pause
%
% $\bullet$ \ \ $P(B|A) = 1$ (modus ponens) \\
% $\bullet$ \ \ $P(\neg A| \neg B) = 1$ (modus tollens) \\
% \pause
% $\bullet$ \ \ $P(B| \neg A) \leq P(B) $ ($A$ falso implica $B$ menos plausible)
% $\bullet$ \ \ $P(A|B) \geq P(A) $ ($B$ verdadero implica $A$ más plausible)
%
%  \vspace{0.6cm}
%
% Dado \ \ $P(B|A) \geq P(B)$ ($A$ verdadero, luego $B$ más plausible) \\
%
% \pause
%
% $\bullet$ \ \ $P(B|\neg A) \leq P(B)$ ($A$ falso, luego $B$ menos plausible) \\
% $\bullet$ \ \ $P(A|B) \geq P(A)$ ($B$ verdadero, luego $A$ más plausible) \\
% $\bullet$ \ \ $P(\neg A|\neg B) \geq P(\neg A)$ ($B$ falso, luego $A$ menos plausible)
%
%
% \end{frame}
%
%
% \begin{frame}[plain]
% \begin{textblock}{160}(0,4)
%  \centering \LARGE El problema de la lógica extendida \\
%  \large La complejidad en memoria
%  \end{textblock}
%  \vspace{1.5cm}
%
%  Tenemos $n=27$ variables, las letras del alfabeto \\
%
%  \pause
%
%  \vspace{0.5cm}
%
%  $\bullet$ En lógica binaria asignamos valores a las variables, y con las reglas de la lógica calculamos las combinaciones. \pause Con 27 bits nos alcanza.  \\
%
%  \vspace{0.5cm}
%
%  \pause
%
%  $\bullet$ En lógica extendida debemos asignar una probabilidad a cada posible combinación de todos los posibles combinaciones de valores de verdad.  \\
%
%  \vspace{0.2cm}
%   \pause
%
%   Necesitamos guardar $2^{27}-1$ números reales  \\[0.2cm]
%
%  \Wider[-2cm]{
%  \begin{itemize}
%  \item[$1$:] $P(A, B, \dots, Z) =  \dots$ \\
%  \item[$.$:] $\dots$ \\
%  \item[$2^{27}-1$:] $P(\neg A, \neg B, \dots, Z) =  \dots$ \\
%  \item[$2^{27}$:] $P(\neg A, \neg B, \dots, \neg Z) =  1 - \sum P(\dots)$
%  \end{itemize}
%  }
% \end{frame}
%



\begin{frame}[plain]
\begin{textblock}{160}(0,4)
\centering \LARGE El costo computacional \\
\large Problema histórico de la probabilidad
\end{textblock}


\only<2->{
\begin{textblock}{160}(0,22) \centering
\Large La \textbf{aplicación estricta} de las reglas de la probabilidad \\

obligan a \textbf{evaluar todo el espacio de hipótesis}.

\end{textblock}
}

\begin{textblock}{120}(20,50)
\only<3->{$\bullet$ Siglo 18: Nace la probabilidad}

\only<4->{$\bullet$ Siglo 19: Física estadística}

\only<5->{$\bullet$ Siglo 20: Estimadores puntuales (evitan evaluar todo el espacio)}

\only<6->{$\bullet$ Siglo 21: Aproximación de la inferencia exactas (de todo el espacio)}
\end{textblock}

\end{frame}



\begin{frame}[plain]
\begin{textblock}{160}(0,4)
\centering  \LARGE Modelos lineales \\[-0.1cm]
 \large en las hipótesis
\end{textblock}


\begin{textblock}{150}(0,22)
\begin{equation*}
\begin{split}
f(\text{dato} = x,\text{hipótesis}= \bm{w})
&= \sum_{i=0}^{\texttt{len(}\bm{w}\texttt{)}-1} w_i \cdot x^i \\[0.6cm]
p(y | x, \bm{w}, \beta ) &= \N(y \,|\, f(x,\bm{w}), \beta^2) \\[0.6cm]
\onslide<2->{
p(w_i) &= \N(w_i \,|\, 0, \sigma_{i}^2) \\[0cm]}
\end{split}
\end{equation*}
\end{textblock}

\end{frame}


\begin{frame}[plain]
\begin{textblock}{160}(0,4)
\centering \LARGE Modelos lineales
\end{textblock}
 \vspace{1.25cm}

\only<1->{
\begin{textblock}{160}(0,16)\centering
\ \ Función objetivo \\

$\N(y \,| \, \text{sen}(x), \beta^2) \ \ x \in [-\pi,\pi]$ \\[0cm]

       \includegraphics[width=0.40\textwidth]{figuras/pdf/model_selection_true_and_sample}
\end{textblock}
}

\only<2>{
\begin{textblock}{160}(0,78) \LARGE \centering
¿Cuál es el mejor modelo lineal?
\end{textblock}
}

\end{frame}


\begin{frame}[plain]
\begin{textblock}{160}(0,4)
\centering \LARGE Siglo 20: Estimadores puntuales \\
\large \sout{Evaluación} Selección de hipótesis
\end{textblock}


\only<1-10>{
\begin{textblock}{160}(0,14) \centering
\begin{equation*}
 \underset{\bm{w}}{\text{ max }} P(\bm{y} | \bm{x}, \bm{w}, \beta) = \underset{\bm{w}}{\text{ min }} \sum_{i=1}^{n}  (y_i - f(x_i, \bm{w}) )^2
\end{equation*}
\end{textblock}
}


\only<11>{
\begin{textblock}{160}(0,20) \centering
\Large Perio si empezamos a ver datos $x \notin [-\pi,\pi]$ este modelo no sirve!
\end{textblock}
}


\begin{textblock}{80}(0,34)\centering
\only<2-3>{\includegraphics[width=0.9\textwidth]{figuras/pdf/model_selection_OLS.pdf}}\only<4>{\includegraphics[width=0.9\textwidth]{figuras/pdf/model_selection_OLS_best-at-train.pdf}}
\end{textblock}



\only<3-4>{
\begin{textblock}{80}(80,35)\centering
\includegraphics[width=0.9\textwidth]{figuras/pdf/model_selection_maxLikelihood.pdf}
\end{textblock}
}


\only<5-9>{
\begin{textblock}{140}(10,36)\centering
\begin{align*}
P(\text{dato}|\text{Modelo}) & = \only<6->{\phantom}{\sum_{\text{hipótesis}}}  P(\text{dato} \, | \, \text{hipótesis}, \text{Modelo}) \only<6->{\phantom}{P(\text{hipótesis} | \text{Modelo})} \\
\only<7->{& = P(\text{dato} \, | \, \overbrace{\underset{h}{\text{arg max}} \ P(\text{dato}|h, \text{Modelo})}^{\text{Hipótesis que mejor predice}},  \, \text{Modelo} )} \\[0.5cm]
\onslide<9>{P(\text{dato}_{\textbf{Testear}}|\text{Modelo}) & = P(\text{dato}_{\textbf{Testear}} \, | \, \underset{h}{\text{arg max}} \ P(\text{dato}_{\textbf{Entrenar}}|h, \text{Modelo}),  \, \text{Modelo} )}
\end{align*}

\onslide<9>{Testeo y Entrenamiento}

\end{textblock}
}


\only<8>{
\begin{textblock}{140}(10,72)\centering
\Large

¿Predecimos o ``post-decimos''?
\end{textblock}
}



\only<10->{
\begin{textblock}{160}(0,33.5)\centering
\phantom{y} Con testeo y entrenamiento

\includegraphics[width=0.43\textwidth]{figuras/pdf/model_selection_OLS_best-at-test.pdf} \hspace{0.4cm}
\onslide<10->{\includegraphics[width=0.42\textwidth]{figuras/pdf/model_selection_maxApriori_online.pdf}}
\end{textblock}
}


\end{frame}


\begin{frame}[plain]

\centering
\LARGE

\vspace{0.5cm}

¿Será que la aplicación estricta de las reglas de la \\

probabilidad produce sobreajuste (\textit{overfitting})?

\pause \vspace{1cm}

¿Y entonces el sistema de razonamiento en \\

contextos de incertidumbre funciona mal?



\end{frame}


\begin{frame}[plain]
\begin{textblock}{160}(0,4)
\centering  \LARGE Siglo 21: Inferencia exacta \\
\large Aplicación estricta de las reglas de la probabilidad
\end{textblock}

\only<2->{
\begin{textblock}{140}(10,26)\centering
\Large El modelo lineal tiene solución analítica! \\
\end{textblock}
}

\only<3->{
\begin{textblock}{140}(10,42)
\large $\bullet$  Para evaluar las hipótesis al interior de los modelos (\textbf{posterior}),
\begin{equation*} \Large
P(\text{Hipótesis}|\text{Datos},\text{Modelo})
\end{equation*} \\[0.5cm]
\only<4->{\large $\bullet$ Y para evaluar modelos alternativos (\textbf{evidencia}),
\begin{equation*} \Large
P(\text{Modelo}|\text{Datos})
\end{equation*}}
\end{textblock}
}

\end{frame}



\begin{frame}[plain]

\begin{textblock}{60}(0,14)
\begin{equation*}
P(\text{Hipótesis}|\text{Datos},\text{Modelo})
\end{equation*} \\[0.5cm]
\end{textblock}


\Wider[-3cm]{
 \begin{figure}
\begin{subfigure}[t]{0.32\textwidth}
\onslide<3->{\caption*{Verosimilitud}}
\end{subfigure}
\begin{subfigure}[t]{0.32\textwidth}
\caption*{Priori\onslide<3->{/Posteriori}}
\includegraphics[width=\textwidth]{figuras/pdf/linearRegression_posterior_0.pdf}
\end{subfigure}
\begin{subfigure}[t]{0.32\textwidth}
\onslide<2->{
\caption*{Espacio de datos}
\includegraphics[width=\textwidth]{figuras/pdf/linearRegression_dataSpace_0.pdf}}
\end{subfigure}


\begin{subfigure}[c]{0.32\textwidth}
\onslide<3->{\includegraphics[width=\textwidth]{figuras/pdf/linearRegression_likelihood_1.pdf}}
\end{subfigure}
\begin{subfigure}[c]{0.32\textwidth}
\onslide<3->{\includegraphics[width=\textwidth]{figuras/pdf/linearRegression_posterior_1.pdf}}
\end{subfigure}
\begin{subfigure}[c]{0.32\textwidth}
\onslide<3->{\includegraphics[width=\textwidth]{figuras/pdf/linearRegression_dataSpace_1.pdf}}
\end{subfigure}

\begin{subfigure}[c]{0.32\textwidth}
\onslide<4->{\includegraphics[width=\textwidth]{figuras/pdf/linearRegression_likelihood_2.pdf}}
\end{subfigure}
\begin{subfigure}[c]{0.32\textwidth}
\onslide<4->{\includegraphics[width=\textwidth]{figuras/pdf/linearRegression_posterior_2.pdf}}
\end{subfigure}
\begin{subfigure}[c]{0.32\textwidth}
\onslide<4->{\includegraphics[width=\textwidth]{figuras/pdf/linearRegression_dataSpace_2.pdf}}
\end{subfigure}

\end{figure}
}
\end{frame}




\begin{frame}[plain]
\begin{textblock}{160}(0,4)
\centering \LARGE La función de costo epistémica \\
\end{textblock}


\begin{textblock}{160}(0,22) \centering
\Large Todos los datos son de testeo y entrenamiento:
\large
\begin{equation*}
\underbrace{P(\text{\En{Data}\Es{Datos}} = \{d_1, d_2, \dots \}|\text{Modelo})}_{\text{\small Evidencia: predicción del modelo}}  =  \underbrace{P(d_1 |\text{Modelo})}_{\text{\small Predic\En{tion}\Es{ción} 1}} \, \underbrace{P(d_2 | d_1 , \text{Modelo})}_{\text{\small Predic\En{tion}\Es{ción} 2}} \dots
\end{equation*}
\end{textblock}

\only<2->{
\begin{textblock}{140}(10,56)\centering
\Large La predicción se hace con todas las hipótesis \large
\begin{align*}
P(\text{dato}_1|\text{Modelo}) & = \sum_{\text{hipótesis}}  P(\text{dato}_1| \text{hipótesis}, \text{Modelo}) P(\text{hipótesis} | \text{Modelo}) \\[0.2cm]
%\onslide<3>{P(\text{Datos}) & = \sum_{\text{Modelo}} P(\text{Datos}\,| \text{Modelo}) \, P(\text{Modelo})}
\end{align*}
\end{textblock}
}


\end{frame}



\begin{frame}[plain]
\begin{textblock}{160}(0,4)
\centering  \LARGE Siglo 21: Inferencia exacta \\
\large Aplicación estricta de las reglas de la probabilidad
\end{textblock}

\begin{textblock}{80}(80,22)\Large
\begin{equation*}
P(\text{Modelo}|\text{Datos})
\end{equation*}
\end{textblock}


\begin{textblock}{160}(0,32)
     \centering
       \includegraphics[width=0.45\textwidth]{figuras/pdf/model_selection_MAP_non-informative}
       \includegraphics[width=0.445\textwidth]{figuras/pdf/model_selection_evidence}
\end{textblock}

\end{frame}



\begin{frame}[plain]
\begin{textblock}{160}(0,4)
\centering \LARGE  Evidencia \\
\large Balance natural entre complejidad y predicci\'on
\end{textblock}


 \begin{textblock}{120}(20,12)
  \centering
  \includegraphics[width=0.9\textwidth]{figuras/pdf/evidencia_de_modelos_alternativos}
 \end{textblock}

  \end{frame}



\begin{frame}[plain]
\begin{textblock}{160}(0,4)
\centering \LARGE  La verdadera predicción \\
\large Predicción con la contribución de todos los modelos
\end{textblock}


\begin{textblock}{160}(0,21)
\begin{equation*}
P(\text{Datos}) =  \sum_{\text{Modelo}} P(\text{Datos}|\text{Modelo}) P(\text{Modelo})
\end{equation*}
\end{textblock}

%
% \begin{textblock}{160}(0,35)\large
% \begin{equation*}
% P(\text{Modelo}|\text{Datos})
% \end{equation*}
% \end{textblock}


\only<2>{
\begin{textblock}{70}(5,50)
     \centering

     Si aparecen datos $x \notin [-\pi, \pi]$ van a poder ser explicados con los modelos más complejos
\end{textblock}


\begin{textblock}{80}(80,40)
     \centering
       \includegraphics[width=0.8\textwidth]{figuras/pdf/model_selection_evidence} \hfill
\end{textblock}
}

\end{frame}


\begin{frame}[plain]
\begin{textblock}{160}(0,4)
\centering \LARGE Limitaciones \\
\large Modelo lineal
\end{textblock}


\begin{textblock}{160}(0,24) \Large \centering
La transformaciones no-lineales $\Phi(x)$ están fijas.
\begin{equation*}
f(x,\bm{w}) = \bm{w}^T \Phi(x)
\end{equation*}
\end{textblock}

\only<2->{
\begin{textblock}{160}(0,56) \Large \centering
Necesitaríamos infinitos modelos \\
para poder representar cualquier función!
\end{textblock}
}

\end{frame}


\begin{frame}[plain]
\begin{textblock}{160}(0,4)
\centering \LARGE Procesos Gaussianos \\
\large Evaluación de infinitos modelos $f(\bm{x})$
\end{textblock}
\vspace{1cm}\centering

\begin{equation*}
\begin{split}
p(y_i) &= \N(y_i|f(\bm{x}_i), \beta^2) \ \ \ \ \text{ donde } f \text{ es la función objetivo}\\[0.6cm]
&\text{Prior sobre los modelos:}\\
 f(\bm{x}) &\sim \mathcal{GP}(m(\bm{x}), k(\bm{x},\bm{x}^{\prime}) )  \\[0.4cm]
\onslide<2->{& m(\bm{x})  = \mathbb{E}[f(\bm{x})] \\
& k(\bm{x},\bm{x}^{\prime}) = \mathbb{E}[(f(\bm{x})-m(\bm{x}))(f(\bm{x}^{\prime})-m(\bm{x}^{\prime}))]}
\end{split}
\end{equation*}

\Large \vspace{0.6cm}

\onslide<3->{Posterior sobre los modelos tiene solución analítica!}

\end{frame}


\begin{frame}[plain]
\begin{textblock}{160}(0,4)
\centering \LARGE Procesos Gaussianos \\
\large Evaluación de infinitos modelos $f(\bm{x})$
\end{textblock}
\centering
\vspace{1.5cm}

\includegraphics[width=0.8\textwidth]{figuras/pdf/gaussianProcess.pdf}


\end{frame}



\begin{frame}[plain]
\begin{textblock}{160}(0,4)
\centering \LARGE Procesos Gaussianos \\
\large y Redes Neuronales
\end{textblock}


\begin{textblock}{140}(10,26)
\only<2->{
\Large Se han demostrado las siguientes equivalencia} \large

\vspace{0.6cm}

\only<3->{
\hspace{0.25cm} $\bullet$ Neal, RM. 1994. \textit{Priors for infinite networks}. Universidad de Toronto.

\normalsize Equivalencia entre una red neuronal de una sola capa totalmente conectada infinitamente ancha y los proceso gaussiano.}

\vspace{0.6cm}

\only<4->{
\hspace{0.25cm} \large $\bullet$ Lee et al (Google Brain). 2018. \textit{Deep Neural Networks as Gaussian Process.} International Conference on Learning Representations

\hspace{0.5cm} \normalsize Derivan la equivalencia exacta entre redes infinitamente profundas y GPs.}
\end{textblock}


\end{frame}

\begin{frame}[plain,noframenumbering]
\centering \vspace{0.5cm}
\includegraphics[width=1\textwidth]{../../auxiliar/static/BP.png}
\end{frame}


%
% \begin{frame}[plain]
% \begin{textblock}{160}(0,4)
% \centering \LARGE Regularizadores L2 \\
% \large para estimación puntual\\
% \end{textblock}
%
% \only<2->{
% \begin{textblock}{160}(0,12) \centering
% \begin{equation*}
%  \underset{\bm{w}}{\text{ max }} P(\bm{y} | \bm{x}, \bm{w}, \beta) P(\bm{w}) = \underset{\bm{w}}{\text{ min }} \sum_{i=1}^{n}  (y_i - f(x_i, \bm{w}) )^2 + \sum_{j=0}^{||\bm{w}||-1}  (0 - w_j)^2
% \end{equation*}
% \end{textblock}
% }
%
% \only<3->{
% \begin{textblock}{160}(0,34)
%      \centering
%        \includegraphics[width=0.45\textwidth]{figuras/pdf/regularizador} \hfill
% \end{textblock}
% }
%
% \end{frame}
%
%
%
%
%
%
%
% %
% % \begin{frame}[plain]
% % \begin{textblock}{96}(0,6.5)\centering
% % {\transparent{0.9}\includegraphics[width=0.8\textwidth]{../../../aux/static/inti.png}}
% % \end{textblock}
% %
% % \begin{textblock}{160}(96,5.5)
% % \includegraphics[width=0.35\textwidth]{../../../aux/static/pachacuteckoricancha}
% % \end{textblock}
% % \end{frame}
%



\end{document}



